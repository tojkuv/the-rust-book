%
% LaTeX Problem Set Template originally designed
% by former CS103 TA Sachin Padmanabhan, with updates,
% edits, and simplifications by the lovely folks below.
%
% Updated for Fall 2018 by Michael Zhu
% Updated for Fall 2019 by Joshua Spayd
% Updated for Fall 2020 by Lucy Lu
% Updated for Winter/Spring 2021 by Cynthia Bailey Lee
% Updated for Fall 2021 by Grant McClearn
% Commands slimmed down and simplified in Fall 2022 by Keith Schwarz

\documentclass{article}
\usepackage{amsmath}
\usepackage{amssymb}
\usepackage{amsthm}
\usepackage{amssymb}
\usepackage{mathdots}
\usepackage{braket}
\usepackage[pdftex]{graphicx}
\usepackage{fancyhdr}
\usepackage[margin=1in]{geometry}
\usepackage{multicol}
\usepackage{bm}
\usepackage{listings}
\PassOptionsToPackage{usenames,dvipsnames}{color}  %% Allow color names
\usepackage{pdfpages}
\usepackage{algpseudocode}
\usepackage{tikz}
\usepackage{enumitem}
\usepackage[T1]{fontenc}
\usepackage{inconsolata}
\usepackage{framed}
\usepackage{wasysym}
\usepackage[thinlines]{easytable}
\usepackage{hyperref}
\usepackage{wrapfig}
\hypersetup{
    colorlinks=true,
    linkcolor=blue,
    filecolor=magenta,
    urlcolor=blue,
}

\title{CS 103: Mathematical Foundations of Computing\\Problem Set \#1}
\author{[TODO: Replace this with your name(s)]}
\date{\today}

% Running author based on https://tex.stackexchange.com/questions/68308/how-to-add-running-title-and-author#answer-68310
\makeatletter
\let\runauthor\@author
\makeatother

\lhead{\runauthor}
\chead{Problem Set 1}
\rhead{\today}
\lfoot{}
\cfoot{CS 103: Mathematical Foundations of Computing --- Fall 2022}
\rfoot{\thepage}

\newcommand{\abs}[1]{\lvert #1 \rvert}
\renewcommand{\(}{\left(}
\renewcommand{\)}{\right)}
\newcommand{\floor}[1]{\left\lfloor#1\right\rfloor}
\newcommand{\ceil}[1]{\left\lceil#1\right\rceil}
\newcommand{\pd}[1]{\frac{\partial}{\partial #1}}
\newcommand{\powerset}[1]{\wp\left(#1\right)}
\newcommand{\suchthat}{\ \vert \ }
\newcommand{\naturals}{\mathbb{N}}
\newcommand{\integers}{\mathbb{Z}}
\newcommand{\reals}{\mathbb{R}}
\renewcommand{\qed}{\blacksquare}
\newcommand{\accepts}{\text{ accepts }}
\newcommand{\rejects}{\text{ rejects }}
\newcommand{\loopson}{\text{ loops on }}
\newcommand{\haltson}{\text{ halts on }}
\newcommand{\encoded}[1]{\left\langle#1\right\rangle}
\newcommand{\rlangs}{\mathbf{R}}
\newcommand{\relangs}{\mathbf{RE}}
\newcommand{\corelangs}{\text{co-}\mathbf{RE}}
\newcommand{\plangs}{\mathbf{P}}
\newcommand{\nplangs}{\mathbf{NP}}

\renewcommand{\labelitemii}{$\bullet$}
\renewcommand\qedsymbol{$\blacksquare$}
\newenvironment{prf}{{\bfseries Proof.}}{\qedsymbol}
\renewcommand{\emph}[1]{\textit{\textbf{#1}}}
\newcommand{\annotate}[1]{\textit{\textcolor{blue}{#1}}}
\usepackage{mdframed}
\usepackage{float}

\makeatother

\definecolor{shadecolor}{gray}{0.95}

\theoremstyle{plain}
\newtheorem*{lem}{Lemma}

\theoremstyle{plain}
\newtheorem*{claim}{Claim}

\theoremstyle{definition}
\newtheorem*{answer}{Answer}

\newtheorem{theorem}{Theorem}[section]
\newtheorem*{thm}{Theorem}
\newtheorem{corollary}{Corollary}[theorem]
\newtheorem{lemma}[theorem]{Lemma}

\renewcommand{\headrulewidth}{0.4pt}
\renewcommand{\footrulewidth}{0.4pt}

\setlength{\parindent}{0pt}

\pagestyle{fancy}

\renewcommand{\thefootnote}{\fnsymbol{footnote}}

\usepackage{boxedminipage}
\newenvironment{blank}{\colorbox{shadecolor}{\strut \underline{\ \ \ \ \ }}}

\begin{document}

\maketitle

\begin{center}
\emph{Due Friday, October 7 at 2:30 pm Pacific}
\end{center}

\section*{Symbols Reference}
Here are some symbols that may be useful for this problem set.

\begin{itemize}
\item Empty set: $\emptyset$
\item Power set: $\powerset{S}$
\item Set of Natural numbers: $\naturals$
\item Union, intersection: $\cup$, $\cap$
\item Equal, not equal: $x = x$, $x \neq y$
\item Element-of, not element-of: $x \in S$, $y \not \in S$
\item Subset-of, not subset-of: $A \subseteq B$, $A \not \subseteq C$
\item Symmetric difference: $S \triangle T$
\item Modular congruence: $x \equiv_k y$
\item Sets: $\Set{1, 2, 3}$, $\Set{n \in \naturals \suchthat n \text{ is even}}$
\end{itemize}

LaTeX typing tips:
\begin{itemize}
\item Set (curly braces need an escape character backslash): ${1, 2, 3}$ (incorrect), $\{1, 2, 3\}$ or $\Set{1, 2, 3}$ (correct)
\item Exponents (use curly braces if exponent is more than 1 character): $x^2$, $2^{3x}$
\item Subscripts (use curly braces if subscript is more than 1 character): $x_0$, $x_{10}$
\end{itemize}

\pagebreak

Problems One and Two are to be answered by editing the appropriate files
(\texttt{MuchAdoAboutNothing.sets} and \texttt{SetTheory.cpp}, respectively).
Do not put your answers to Problems One and Two in this file.

\newpage

\section*{Problem Three: Describing the World in Set Theory}

i.
\begin{shaded}
$$C \subseteq A\& C \subseteq B \& D \subseteq A \& D \subseteq B$$
\end{shaded}

ii.
\begin{shaded}
$$\exists x x \in Y \| x \notin D$$
\end{shaded}

iii.
\begin{shaded}
$$\exists (x,y) ((x,y) \in (\wp(S) \cup \wp(F)) \| x \in \wp(S), y \in \wp(F))$$
\end{shaded}

\newpage

\section*{Problem Four: Writing Direct Proofs}
i.
\begin{shaded}
pick any integer $n$ such that $n$ is odd
\end{shaded}

ii.
\begin{shaded}
we want to show that $n^2$ is odd
\end{shaded}

iii. Fill in the blanks to Problem Four, part iii below.
\begin{center}
\begin{boxedminipage}{0.9\textwidth}
\emph{Theorem:} For any integer $n$, if $n$ is odd, then $n^2$ is odd. \\
\emph{Proof: } Pick any integer for $n$ that is odd. We want to show that $n"2$ is odd. Since $n$ is odd, there is an integer $k$ where $n = 2k + 1$.
\annotate{(Note: When following the direct proof guidelines, the first three sentences are fairly mechanical, but now we need to exercise actual creativity. It helps to look at our want-to-show to remind ourselves of our destination on this journey. It says we want to show something about $n^2$, so we will begin with the expression $n^2$, and then apply algebra to it.)}
Then we see that
\begin{equation*}
\begin{split}
n^{2} &=  (2m + 1)^2\\
&=  4k^2 + 4k + 1 \\
&=  2(2k^2 + 2k) + 1.
\end{split}
\end{equation*}
Therefore, there is an integer $m$ (namely, $m = 2k^2 + 2k$) such that $n^2 = 2m + 1$,
so $n^2$ is odd \annotate{(Write the ``Want to Show'' fact here to announce that you
reached it.)}, as required. \qedsymbol
\end{boxedminipage}
\end{center}
iv.
\begin{shaded}
\textbf\textit{Theorem:} For any integer $n$, if $n$ is odd, then $n^2$ is odd.
\textbf\textit{Proof:} Pick any integer $n$ such that $n$ is odd. We want to show that $n^2$ is
odd. Since $n$ is odd, there is an integer $k$ where $n = 2k + 1$
\begin{equation*}
\begin{split}
n^{2} &=  (2m + 1)^2\\
&=  4k^2 + 4k + 1 \\
&=  2(2k^2 + 2k) + 1.
\end{split}
\end{equation*}
Therefore, there is an integer $m$ (namely, $m = 2k^2 + 2k$) such that $n^2 = 2m + 1$,
so $n^2$ is odd, as required. \qedsymbol
\end{shaded}

\newpage

\section*{Problem Five: Writing Proofs by Contrapositive}
\begin{shaded}
the contrapositive of the implication \textbf{if P is true, then Q is true} is the implication
\textbf{if P is false then Q is false}. the contrapositive of an implication makes the same universal
claim as the implication itself. this property of logical reasoning can be used to formulate
contrapositive statements that prove the implication.
\end{shaded}

i.
\begin{shaded}
\textbf\textit{Theorem:} For all integers $a, b, c$, if all $a, b, and c$ are odd,
then $a^2 + b^2 \ne c^2$
\end{shaded}

ii.
\begin{shaded}
if all $a, b, and c$ are odd
\end{shaded}

iii.
\begin{shaded}
then $a^2 + b^2 \ne c^2$
\end{shaded}

iv. Fill in the blanks to Problem Five, part iv. below.
\begin{center}
\begin{boxedminipage}{0.9\textwidth}
\emph{Theorem:} For all integers $a, b, $ and $c$, if $a^{2} + b^{2} = c^{2}$, then at least one of $a, b, $ and $c$ is even. \\
\emph{Proof: } We will prove the contrapositive of this statement, namely, for all integers $a, b, c$, if all $a, b, and c$ are odd,
then $a^2 + b^2 \ne c^2$.  To do so, pick any odd integers for $a, b, c$. We want to show that $a^2 + b^2 \ne c^2$. \\
Since $a, b, $ and c are odd, we know by our result from the previous problem that $a^{2}, b^{2}, $ and $c^{2}$ are odd.\\
Because $a^{2}$ and $b^{2}$ are odd, there exist integers $p$ and $q$ such that $a^2 = p$ and $b^2 = q$. This means that
$a^{2} + b^{2} = $ p + q $=$ an even number, which means that $a^{2} + b^{2}$ is even.\\
However, as mentioned earlier we know that $c^{2}$ is odd. Therefore, we see that $a^2 + b^2 \ne c^2$ \annotate{(Write the ``Want to Show'' fact here to announce that you reached it.)} as required.\qedsymbol
\end{boxedminipage}
\end{center}

v.
\begin{shaded}

\textit{don't repeat definitions, used them instead}
\end{shaded}

\newpage
\section*{Problem Six: Writing Proofs by Contradiction}

i.
\begin{shaded}
Write your answer to Problem Six, part i. here.
\end{shaded}

ii. Fill in the blanks to Problem Six, part ii. below.
\begin{center}
\begin{boxedminipage}{0.9\textwidth}
\emph{Theorem:} For all integers $m$ and $n$, if $mn$ is even and $m$ is odd, then $n$ is even. \\
\emph{Proof: } Assume for the sake of contradiction that \blank. Since $m$ is \blank, we know that there is an integer $k$ where \blank. Similarly, since $n$ is \blank, there is an integer $r$ where \blank. Then we see that
\begin{equation*}
\begin{split}
mn &= \blank \\
&= \blank \\
&= \blank
\end{split}
\end{equation*}
which means that $mn$ is \blank, but this is impossible because \blank.\\
We have reached a contradiction, so our assumption must have been wrong. Therefore, if $mn$ is even and $m$ is odd, then $n$ is even.\qedsymbol
\end{boxedminipage}
\end{center}

\newpage

\section*{Problem Seven: Proving Existentially-Quantified Statements}

i. Fill in the blanks to Problem Seven, part i. below.
\begin{center}
\begin{boxedminipage}{0.9\textwidth}
\emph{Theorem:} There are real numbers $a$ and $b$ where $\floor{a} \cdot \ceil{b} \not= \floor{ab}$. \\ \emph{Proof: } Pick $a = $ \blank and $b = $ \blank. Then we see that
\begin{equation*} \floor{a} \cdot \ceil{b} =
\floor{\blank} \cdot \ceil{\blank} =
\blank \cdot \blank =
\blank, \end{equation*}
but
\begin{equation*} \floor{ab} = \floor{\blank \cdot \blank} = \floor{\blank} = \blank. \end{equation*}
Thus \blank $\not=$ \blank, as required. \qedsymbol
\end{boxedminipage}
\end{center}

ii. Fill in the blanks to Problem Seven, part ii. below.
\begin{center}
\begin{boxedminipage}{0.9\textwidth}
\emph{Theorem:} There exist natural numbers $a, b, c,$ and $d$ such that $a > b > c > d > 0$ and $a^2 + b^2 + c^2 + d^2 = 137.$\\
\emph{Proof: } Pick $a = $ \blank, $b = $ \blank, $c = $ \blank, and $d = $ \blank. Then we see that
\begin{equation*} \blank > \blank > \blank > \blank > \blank \end{equation*}
and
\begin{equation*}
\begin{aligned}
\blank^2 + \blank^2 + \blank^2 + \blank^2
&= \blank^2 + \blank^2 + \blank^2 + \blank^2 \\
&= \blank + \blank + \blank + \blank \\ &= 137, \end{aligned} \end{equation*}
as required. \qedsymbol
\end{boxedminipage}
\end{center}

\newpage

\section*{Problem Eight: Proving Mixed Universal and Existential Statements}
i. Fill in the blanks to Problem Eight, part i. below.

\begin{center}
\begin{boxedminipage}{0.9\textwidth}
\emph{Theorem:} For all integers $x$ and $y$ and any integer $k$, if $x \equiv_{k} y$, then $y \equiv_{k} x$. \\
\emph{Proof: } Let $x, y, $ and $k$ be arbitrary integers where $x \equiv_{k} y$. We want to show that $y \equiv_{k} x$.\\
To do so, we will show that there is an integer $q$ where \blank. \annotate{(Notice that after stating our want-to-show as usual for a direct proof, we expanded the want-to-show to be more specific about what it means to show that. Now our want-to-show has the form of an existential (``...there exists an integer $q$...''). This is an important structural trait to focus in on. It means that in the body of the proof, we are going to have to demonstrate that such a $q$ exists very concretely by proposing a specific value for $q$.)}
Because $x \equiv_{k} y$, we know there is an integer $r$ such that \blank.\\
Now, let $q = $ \blank. \annotate{(Here it is, our announcement of the value we are proposing for $q$. Next, we need to justify to the reader that our choice works. What follows is that justification. Always do it in this order: first announce the value, then justify.)} Then we see that
\begin{equation*}
\begin{split}
y &= \blank \annotate{(Do not write x + qk here)}\\
&= \blank \\
&= x + qk
\end{split}
\end{equation*}
which is what we needed to show. \qedsymbol
\end{boxedminipage}
\end{center}

ii.
\begin{shaded}
Write your answer to Problem Eight, part ii. here.
\end{shaded}

iii.
\begin{shaded}
Write your answer to Problem Eight, part iii. here.
\end{shaded}

\newpage

\section*{Problem Nine: Proof Critiques}
i.
\begin{shaded}
Write your answer to Problem Nine, part i. here.
\end{shaded}

ii.
\begin{shaded}
Write your answer to Problem Nine, part ii. here.
\end{shaded}

iii.
\begin{shaded}
Write your answer to Problem Nine, part iii. here.
\end{shaded}

\newpage

\section*{Problem Ten: All Squared Away}

i.

\begin{shaded}

\begin{itemize}
\item When $n = 3$, we can write $n = 2 \cdot \blank + 1$, and one $m$ that works is \blank.
\item When $n = 5$, we can write $n = 2 \cdot \blank + 1$, and one $m$ that works is \blank.
\item When $n = 7$, we can write $n = 2 \cdot \blank + 1$, and one $m$ that works is \blank.
\item When $n = 9$, we can write $n = 2 \cdot \blank + 1$, and one $m$ that works is 16.
\item When $n = 11$, we can write $n = 2 \cdot \blank + 1$, and one $m$ that works is \blank.
\end{itemize}

\end{shaded}

ii.
\begin{shaded}

When $n = 2k + 1$, pick $m = \blank$.

\end {shaded}

iii.

\begin{shaded}
Write your answer to Problem Ten, Part (iii) here.
\end{shaded}

\newpage

\section*{Problem Eleven: Quarter-Squares}
\begin{shaded}
Write your answer to Problem Eleven here.
\end{shaded}

\newpage

\section*{Optional Fun Problem: Infinite Deviation}
\begin{shaded}
(Optionally) Write your answer to the Optional Fun Problem here.
\end{shaded}

\end{document}
